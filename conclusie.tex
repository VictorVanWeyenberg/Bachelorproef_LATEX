%%=============================================================================
%% Conclusie
%%=============================================================================

\chapter{Conclusie}
\label{ch:conclusie}

% TODO: Trek een duidelijke conclusie, in de vorm van een antwoord op de
% onderzoeksvra(a)g(en). Wat was jouw bijdrage aan het onderzoeksdomein en
% hoe biedt dit meerwaarde aan het vakgebied/doelgroep? 
% Reflecteer kritisch over het resultaat. In Engelse teksten wordt deze sectie
% ``Discussion'' genoemd. Had je deze uitkomst verwacht? Zijn er zaken die nog
% niet duidelijk zijn?
% Heeft het onderzoek geleid tot nieuwe vragen die uitnodigen tot verder 
%onderzoek?

\section{Is het mogelijk om digitaal te gaan?}

Uit sectie \ref{onderzoeksvraag1} bleek dat het wel degelijk mogelijk zou zijn om digitaal te gaan. Er bestaat al technologie voor, het moet enkel nog correct geïmplementeerd worden. De correcte implementatie is hier van essentie, zoals bij de testresultaten van JASS te zien was.

Dat niet alleen. Op dit moment wordt er meer onderzoek gevoerd naar toepassingen van andere methodes van digitale geluidsgeneratie zoals granular en wavelet synthesis. Houthave zei in zijn interview: \textit{``granulaire synthesis is dan weer muzikaler van aard. Daar definieer je instrumenten.''} \autocite{thomashouthave} Dit onderzoek heeft zich beperkt tot subtractive synthesis omdat het de meest intuïtieve generatie- en verwerkingsmethode is voor het doelpubliek. \textcite{granular} legt in zijn paper een methode voor real-time granular synthesis uit. Zoals in sectie \ref{methode:granular} staat moet hier een groot aantal parameters bediend worden dat toeneemt naargelang de complexiteit van het geluid. Daarbij is \textit{simplicity key} voor een product dat aan de consument verkocht wordt. Hier kan zeker verder onderzoek gevoerd worden.

User interfaces vallen buiten de scope van dit onderzoek. Voor verdere ontwikkeling is belangerijk dat de profielen live slechts één parameter op eenzelfde moment bedienen. Met toetsenbord en muis is het zeker haalbaar om de bediening van één parameter uit te voeren.

\section{Is het nuttig om digitaal te gaan?}

Sectie \ref{onderzoeksvraag2} toonde aan dat alle profielen een digitale transitie zouden verwelkomen. Wat opvalt is dat sommige profielen bepaalde eisen stellen voor het eindproduct. Zo eisen muzikanten om het taktiele in hun instrument te behouden en hebben producers nood aan een hogere geluidsresolutie dan standaard CD-kwaliteit. Zolang aan die wensen voldaan kan worden, is er interesse naar digitalisatie op de markt.

Of de consumenten er direct positief op gaan reageren is een ander verhaal. Vincent en Boone zien veel potentieel in digitale verwerking voor live muziek. Voornamelijk omdat digitaal het werkproces versnelt en vergemakkelijkt. Toch blijft Boone sceptisch over de digitale transitie. Ten eerste omdat de genres van zijn klanten niet thuis horen in een digitaal milieu en ten tweede omdat de digitale tafels van vandaag geen deftige geluidsresolutie kunnen verwerken. Er is dus zeker vraag naar digitale mengtafels die hogere geluidsresoluties dan \textit{standaard CD-kwaliteit} \autocite{peterboone} aankunnen.

Uit andere interviews en uit \ref{onderzoeksvraag1} bleek echter dat er geen hoorbaar verschil is tussen digitaal en analoog. Natuurlijk zal analoge verwerking altijd wat ruis en vuil hebben. Sommige artiesten en producers zijn hier net naar opzoek. \textit{``Iemand kan vinden dat digitaal veel cleaner is. Anderen vinden dat ze karakter missen. Maar veel van dat "vuil" kan ook gegenereerd worden door een plugin,''} vertelt Vincent over digitale of analoge voorkeuren. \autocite{bartvincent} 

Boone vertelt na zijn interview over een fout die hij in een track van de Beatles gehoord heeft. Het is algemeen geweten dat er zich veel foutjes in de albums van de Beatles verschuilen. Dat zijn dingen die niet meer kunnen in de muzieksector van vandaag. Een song moet een afgewerkt product zijn, zonder vuil of fouten. Daar kan digitaal zeker bij helpen, zegt Boone. \textit{``Dat is wel een voordeel van digitaal ten opzichte van analoog. Als je bij analoog een te laag opnameniveau hebt en je trekt het op, dan trek je ook de ruis op. En digitaal heeft geen ruis.''} \autocite{peterboone}

Een andere manier om deze vraag te beantwoorden is door te kijken naar de voordelen die digitaal ten opzichte van analoog heeft.

\begin{itemize}
	\item Het opslaan van instellingen en die met een druk op de knop terug kunnen inladen.
	\item Het standaardiseren van die digitale instellingen zodat ze tussen opstellingen en artiesten gedeeld kunnen worden.
	\item De prijs die significant lager ligt in vergelijking met de analoge varianten.
	\item De compactheid van het product.
	\item Digitaal is \textit{cleaner} en bevat geen ruis.
\end{itemize}

Hieruit wordt geconcludeerd dat het algemeen zeker nuttig is om digitaal te gaan. Maar daarmee is niet aan de wensen van ieder profiel voldaan. In sectie \ref{endeontwikkelaars} wordt hier dieper op ingegaan.

\section{Is het realistisch om digitaal te gaan?}
\label{conclusie3}

Om deze vraag te beantwoorden moest slechts één van de libraries een goede performantie vertonen. Pas dan mocht gezegd worden dat het realistisch is om digitaal te gaan. In sectie \ref{onderzoeksvraag3} werd aangetoond dat JSyn alle testcases aankon. Dus wordt geconcludeerd dat het realistisch is om digitaal te gaan.

Voor een open-source Java library heeft JSyn goed gescoord op de empirische test. Maar wanneer puntje bij paaltje komt, zal het eindproduct niet met JSyn geschreven worden. Bedenk dat de empirische tests afgenomen zijn op een particuliere laptop. De laptops verkrijgbaar op de hedendaagse markt zijn veel sterker dan onze testmachine. Bedenk hoe een C++-library zou presteren. Zulke low-level programmeertalen presteren zeker beter dan Java.  

\section{Wat wilt dit zeggen voor ontwikkelaars?}
\label{endeontwikkelaars}

\begin{table}[]
\begin{tabular}{l|l|l|l|}
\cline{3-4}
\multicolumn{2}{l}{\multirow{2}{*}{}}  & \multicolumn{2}{c|}{\textbf{Empirische tests}}                                                                                                                  \\ \cline{3-4} 
\multicolumn{2}{l}{}   & \textbf{Positief} & \textbf{Negatief} \\ \hline
\multicolumn{1}{|l|}{\multirow{2}{*}{\textbf{Emotionele tests}}} & \textbf{Positief} & Betreed de markt.                                                                           & \begin{tabular}[c]{@{}l@{}}Er is interesse maar\\ het is technisch niet mogelijk.\end{tabular} \\ \cline{2-4} 
\multicolumn{1}{|l|}{}                                           & \textbf{Negatief} & \begin{tabular}[c]{@{}l@{}}Het is technisch mogelijk\\ maar weinig vraag naar.\end{tabular} & \begin{tabular}[c]{@{}l@{}}Slecht idee om de markt\\ te betreden.\end{tabular}                 \\ \hline
\end{tabular}
\caption{Uitkomsten van de Emotionele-Empirische test.}
\label{EEtest}
\end{table}

Tabel \ref{EEtest} geeft beter inzicht in de situatie. JSyn scoorde positief op onze empirische tests. In sectie \ref{conclusie3} wordt besproken dat alternatieve manieren van geluidsgeneratie en -verwerking betere prestaties opleveren. Het is technisch zeker mogelijk om de overstap naar digitaal te maken. De emotionele tests, daarentegen, zijn afhankelijk van profiel tot profiel. Waar de digitalisatie gegeerd is voor sound designers, geluidstechnici en sommige muzikanten, is dat niet altijd het geval bij producers en opname artiesten.

Het is technisch zeker mogelijk om de digitale transitie te maken. Uit de interviews bleek bovendien dat daar ook interesse voor is. Dus als hier verdere ontwikkelingen gebeuren - rekening houdend met profielspecifieke requirements - maakt de ontwikkelaar kans in de markt.

\section{Toekomstig Onderzoek}

\subsection{Grotere Bevraging}

Dit onderzoek geeft een duidelijke indicatie dat er ruimte is voor ontwikkelaars in de markt. Een bredere bevraging van de muzieksector is wenselijk om de requirements van de profielen te verfijnen.

\subsection{Proof-of-concept}

Toekomstig onderzoek kan proberen om digitale versies te maken van analoge apparatuur. Denk hierbij aan mengpanelen, voorversterkers, effectpedalen, synthesizers, direct input boxes etc. Bij voorkeur worden deze in een low-level programmeertaal geschreven, onafhankelijk van een library en met een minimalistische user interface.

\subsection{Alternatieve Methodes voor Geluidsgeneratie en -verwerking}

In sectie \ref{sec:methodesgeneratie} werden verschillende methodes voor geluidsgeneratie en -verwerking besproken. Met name granular, wavelet en corpus-based granular synthesis. Toekomstig onderzoek kan op zoek gaan naar efficiente manieren om deze methodes toe te passen in verwerking en generatie van muziek. Dat niet alleen. Thomas Houthave sprak in zijn interview over toepassingen van AI in het simuleren en immiteren van geluid. Hier ziet hij tepassingen voor in sound design voor game engines. \autocite{thomas houthave} Hij sprak ook over \textit{bionische muzikanten} die jammen met een artiest. Het programma speelt in op noten of akkoorden die de muzikant speelt. Zo kan de muzikant begeleid worden in het creëren van zijn muziek.

\iffalse \lipsum[76-80] \fi

