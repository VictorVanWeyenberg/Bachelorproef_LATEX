%%=============================================================================
%% Conclusie
%%=============================================================================

\chapter{Conclusie}
\label{ch:conclusie}

% TODO: Trek een duidelijke conclusie, in de vorm van een antwoord op de
% onderzoeksvra(a)g(en). Wat was jouw bijdrage aan het onderzoeksdomein en
% hoe biedt dit meerwaarde aan het vakgebied/doelgroep? 
% Reflecteer kritisch over het resultaat. In Engelse teksten wordt deze sectie
% ``Discussion'' genoemd. Had je deze uitkomst verwacht? Zijn er zaken die nog
% niet duidelijk zijn?
% Heeft het onderzoek geleid tot nieuwe vragen die uitnodigen tot verder 
%onderzoek?

\section{Is het mogelijk om digitaal te gaan?}

Uit sectie \ref{onderzoeksvraag1} bleek dat het wel degelijk mogelijk zou zijn om digitaal te gaan. Er bestaat al technologie voor, het moet enkel nog correct geïmplementeerd worden. De correcte implementatie is hier van essentie, zoals bij de testresultaten va JASS te zien was.

Dat niet alleen. Op dit moment wordt er meer onderzoek gevoerd naar toepassingen van andere methodes van digitale geluidsgeneratie zoals granular en wavelet synthesis. Houthave zei in zijn interview: \textit{``granulaire synthesis is dan weer muzikaler van aard. Daar definieer je instrumenten.''} \autocite{thomashouthave} Dit onderzoek heeft zich beperkt tot subtractive synthesis omdat het de meest intuïtieve generatie- en verwerkingsmethode is voor het doelpubliek. \textcite{granular} legt in zijn paper een methode voor real-time granular synthesis uit. Zoals in sectie \ref{methode:granular} staat moet hier rekening gehouden worden met een groot aantal parameters dat toeneemt naargelang hoe complex het geluid is. Daarbij is \textit{simplicity key} voor een product dat aan de consument verkocht wordt. Hier kan zeker verder onderzoek gevoerd worden.

De interviewees... parameters

\section{Is het nuttig om digitaal te gaan?}

Sectie \ref{onderzoeksvraag2} toonde aan dat al onze profielen een digitale transitie zouden verwelkomen. Wat opvalt is dat sommige profielen bepaalde eisen stellen voor het eindproduct. Zo eisen muzikanten om het taktiele in hun instrument te behouden en hebben producers nood aan een hogere geluidsresolutie dan standaard CD-kwaliteit. Zolang aan die wensen voldaan kan worden, kan er gesproken worden van vraag naar digitalisatie op de markt.

Gaat een digitaal product ook direct aanslaan? Peter Boone is hier zeer kritisch over. 

Een andere manier om naar deze vraag te beantwoorden is door te kijken naar de voordelen van digitaal ten opzichte van analoog heeft.

\section{Is het realistisch om digitaal te gaan?}

Het slaagcriterium voor de technische tests was dat wanneer slechts één van de libraries 

\section{Wat wilt dit zeggen voor ontwikkelaars?}

\section{Toekomstig Onderzoek}

\subsection{Prototypering}



\begin{itemize}
	\item Prototypering van een digitale mengtafel applicatie.
	\item Doe het niet in Java, doe het in C! Max...
\end{itemize}

\iffalse \lipsum[76-80] \fi

