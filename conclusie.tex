%%=============================================================================
%% Conclusie
%%=============================================================================

\chapter{Conclusie}
\label{ch:conclusie}

% TODO: Trek een duidelijke conclusie, in de vorm van een antwoord op de
% onderzoeksvra(a)g(en). Wat was jouw bijdrage aan het onderzoeksdomein en
% hoe biedt dit meerwaarde aan het vakgebied/doelgroep? 
% Reflecteer kritisch over het resultaat. In Engelse teksten wordt deze sectie
% ``Discussion'' genoemd. Had je deze uitkomst verwacht? Zijn er zaken die nog
% niet duidelijk zijn?
% Heeft het onderzoek geleid tot nieuwe vragen die uitnodigen tot verder 
%onderzoek?

\section{Is het mogelijk om digitaal te gaan?}

Uit sectie \ref{onderzoeksvraag1} bleek dat het wel degelijk mogelijk zou zijn om digitaal te gaan. Er bestaat al technologie voor, het moet enkel nog correct geïmplementeerd worden. De correcte implementatie is hier van essentie, zoals bij de testresultaten van JASS te zien was.

Dat niet alleen. Op dit moment wordt er meer onderzoek gevoerd naar toepassingen van andere methodes van digitale geluidsgeneratie zoals granular en wavelet synthesis. Houthave zei in zijn interview: \textit{``granulaire synthesis is dan weer muzikaler van aard. Daar definieer je instrumenten.''} \autocite{thomashouthave} Dit onderzoek heeft zich beperkt tot subtractive synthesis omdat het de meest intuïtieve generatie- en verwerkingsmethode is voor het doelpubliek. \textcite{granular} legt in zijn paper een methode voor real-time granular synthesis uit. Zoals in sectie \ref{methode:granular} staat moet hier rekening gehouden worden met een groot aantal parameters dat toeneemt naargelang de complexiteit van het geluid. Daarbij is \textit{simplicity key} voor een product dat aan de consument verkocht wordt. Hier kan zeker verder onderzoek gevoerd worden.

User interfaces vallen buiten de scope van dit onderzoek. Toch, voor verder onderzoek, mag aangekaart worden dat de profielen live slechts één parameter op eenzelfde moment bedienen. Met toetsenbord en muis is het zeker haalbaar om de bediening van één parameter te implementeren.

Het is mogelijk om digitaal te gaan.

\section{Is het nuttig om digitaal te gaan?}

Sectie \ref{onderzoeksvraag2} toonde aan dat al onze profielen een digitale transitie zouden verwelkomen. Wat opvalt is dat sommige profielen bepaalde eisen stellen voor het eindproduct. Zo eisen muzikanten om het taktiele in hun instrument te behouden en hebben producers nood aan een hogere geluidsresolutie dan standaard CD-kwaliteit. Zolang aan die wensen voldaan kan worden, is er interesse naar digitalisatie op de markt.

Of de consumenten er direct positief op gaan reageren is een ander verhaal. Vincent en Boone zien veel potentieel in digitale verwerking voor live muziek. Voornamelijk omdat digitaal het werkproces versnelt en vergemakkelijkt. Toch blijft Boone sceptisch over de digitale transitie. Ten eerste omdat de genres van zijn klanten niet thuis horen in een digitaal milieu en ten tweede omdat de digitale tafels van vandaag geen deftige geluidsresolutie kunnen verwerken. Naar digitale mengtafels die hogere geluidsresoluties dan \textit{standaard CD-kwaliteit} \autocite{peterboone} aankunnen is er dus zeker vraag.

Uit andere interviews en uit \ref{onderzoeksvraag1} bleek echter dat er in essentie geen hoorbaar verschil is tussen digitaal en analoog. Natuurlijk zal analoge verwerking altijd wat ruis en vuil hebben. Sommige artiesten en producers zijn hier net naar opzoek. \textit{``Iemand kan vinden dat digitaal veel cleaner is. Anderen vinden dat ze karakter missen. Maar veel van dat "vuil" kan ook gegenereerd worden door een plugin,''} aldus Vincent. \autocite{bartvincent}

Boone vertelt na zijn interview over een fout die hij in een track van de Beatles gehoord heeft. Het is algemeen geweten dat er veel foutjes in de albums van de Beatles gekropen zijn. Dat zijn dingen die niet meer kunnen in de muzieksector van vandaag. Een song moet een afgewerkt product zijn, zonder vuil of fouten. Daar kan digitaal zeker bij helpen, zegt Boone. \textit{``Dat is wel een voordeel van digitaal ten opzichte van analoog. Als je bij analoog een te laag opnameniveau hebt en je trekt het op, dan trek je ook de ruis op. En digitaal heeft geen ruis.''} \autocite{peterboone}

Een andere manier om deze vraag te beantwoorden is door te kijken naar de voordelen van digitaal ten opzichte van analoog heeft.

\begin{itemize}
	\item Het opslaan van instellingen en die met een druk op de knop terug kunnen inladen.
	\item Het standaardiseren van die digitale instellingen zodat ze tussen opstellingen en artiesten gedeeld kunnen worden.
	\item De prijs die significant lager ligt in vergelijking met de analoge varianten.
	\item De compactheid van het product.
	\item Digitaal is \textit{cleaner} en bevat geen ruis.
\end{itemize}

\section{Is het realistisch om digitaal te gaan?}

Om deze vraag te beantwoorden moest slechts één van de libraries een goede performantie vertonen. Pas dan mocht gezegd worden dat het realistisch is om 

\section{Wat wilt dit zeggen voor ontwikkelaars?}

\section{Toekomstig Onderzoek}

\subsection{Prototypering}



\begin{itemize}
	\item Prototypering van een digitale mengtafel applicatie.
	\item Doe het niet in Java, doe het in C! Max...
\end{itemize}

\iffalse \lipsum[76-80] \fi

