%%=============================================================================
%% Samenvatting
%%=============================================================================

% TODO: De "abstract" of samenvatting is een kernachtige (~ 1 blz. voor een
% thesis) synthese van het document.
%
% Deze aspecten moeten zeker aan bod komen:
% - Context: waarom is dit werk belangrijk?
% - Nood: waarom moest dit onderzocht worden?
% - Taak: wat heb je precies gedaan?
% - Object: wat staat in dit document geschreven?
% - Resultaat: wat was het resultaat?
% - Conclusie: wat is/zijn de belangrijkste conclusie(s)?
% - Perspectief: blijven er nog vragen open die in de toekomst nog kunnen
%    onderzocht worden? Wat is een mogelijk vervolg voor jouw onderzoek?
%
% LET OP! Een samenvatting is GEEN voorwoord!

%%---------- Nederlandse samenvatting -----------------------------------------
%
% TODO: Als je je bachelorproef in het Engels schrijft, moet je eerst een
% Nederlandse samenvatting invoegen. Haal daarvoor onderstaande code uit
% commentaar.
% Wie zijn bachelorproef in het Nederlands schrijft, kan dit negeren, de inhoud
% wordt niet in het document ingevoegd.

\IfLanguageName{english}{%
\selectlanguage{dutch}
\chapter*{Samenvatting}
\iffalse \lipsum[1-4] \fi
\selectlanguage{english}
}{}

%%---------- Samenvatting -----------------------------------------------------
% De samenvatting in de hoofdtaal van het document

\chapter*{\IfLanguageName{dutch}{Samenvatting}{Abstract}}

\iffalse \lipsum[1-4] \fi

Dit onderzoek stelt in vraag waarom de muzieksector analoog te werk blijft gaan in een wereld die steeds digitaler wordt. Er werd een vraag vastgesteld naar digitalisering voor de muzieksector. Ontwikkelaars kunnen inspelen op deze vraag en producten of applicaties ontwikkelen die het creatieproces van artiesten en producers vergemakkelijken.

Dit onderzoek voert een tweedelige test uit: de emprische en emotionele test. Voor de emotionele test zijn interviews gehouden met verschillende profielen uit de muziekwereld. Allereerst werd ieder profiel gevraagd naar de opstelling die ze live of in opname gebruiken. Vervolgens somden ze alsook hun wensen op voor potentiële digitale producten. Zo werd per profiel geregistreerd of hun rol in de muziekwereld interesse had in een digitale transitie.\newline Voor de empirische test zijn de muzikale opstellingen van de profielen vertaald naar een digitale omgeving. Voor de generatie en verwerking van het geluid werd gebruik gemaakt van drie sound synthesis libraries. Dit werd ondergebracht in een uitbreidbare testframework waar de libraries performantiemetingen ondergingen voor ieder profiel. Als slechts één library persistent goede metingen behaalde voor alle profielen, zou het op technisch vlak realistisch zijn om de overstap naar digitaal te maken.

Dit werk bespreekt methodes van digitale geluidsgeneratie en -verwerking alsook hun toepassingen. 
