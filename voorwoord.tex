%%=============================================================================
%% Voorwoord
%%=============================================================================

\chapter*{\IfLanguageName{dutch}{Woord vooraf}{Preface}}
\label{ch:voorwoord}

%% TODO:
%% Het voorwoord is het enige deel van de bachelorproef waar je vanuit je
%% eigen standpunt (``ik-vorm'') mag schrijven. Je kan hier bv. motiveren
%% waarom jij het onderwerp wil bespreken.
%% Vergeet ook niet te bedanken wie je geholpen/gesteund/... heeft

Mijn idee voor dit onderzoek begon in academiejaar 2017 - 2018. Het startte uit mijn enorme interesse voor synthesizers. Specifieker de Eurorack standaardisatie in modulaire synthesizers uit 1996. Het ``modulaire'' aan zulke systemen was oorspronkelijk de mindset van het programmeren. Iedere module heeft zijn eigen atomaire rol in het systeem die via \textit{patching} benut kan worden door gelijk welke andere module. Voordien bestonden er al modulaire synthesizers. Helaas hing de synthesist, bij het kopen van een case, vast aan die bepaalde producent. Andere merken ondersteunden dat formaat niet. Het was een branche van muziek waar een grote DIY-stempel op stond, wat soms gevaarlijk was voor het instrument. 

Vandaag de dag is dat niet veranderd, maar het is sterk verminderd. Nadat Dieter Döpfer - stichter van het welbekende \textit{Doepfer Musikelektronik} - in 1996 de eerste synthesizer in het Eurorack formaat lanceerde, was er sprake van het begin van een standaardisatie. Het project was niet alleen succesvol door de goede documentatie en de daaropvolgende lanceringen van Eurorack modules maar ook door de aantrekkelijke form factor.

De kracht van zulke systemen komt het meest naar voor in studio's voor filmproductie. 	De muziek voor grote hedendaagse filmtitels zoals Interstellar en Deadpool zijn gemaakt op Moog System 35. System 35 stamt af uit de jaren '70. Het is gigantisch in vergelijking het Eurorack formaat en veel duurder.

Modulaire systemen zijn in mijn opinie zeer sterke tools voor sound design. Helaas vragen zulke systemen een enorme kost. De eerste stap om zulke systemen toegankelijker te maken is standaardisatie; dat is wat Döpfer in 1996 heeft gedaan. Met een stijgend aanbod en een meer gedeelde marktvraag voor een klein formaat als Eurorack, kon de prijs van de modules snel dalen. Wat is de volgende stap? Er is nu een standaardisatie in het formaat voor de modules. In een wereld die steeds digitaler wordt, missen we nog een manier van \textit{patches} te delen. Kan digitaal de volgende standaardisatie zijn?

Intussen heb ik zelf al meerdere modulaire synthesizer programma's geschreven daar waar ze in mijn ogen productiewaardig leken. Ik vroeg me af waarom niemand dit eerder gedaan heeft. Dat was het oorspronkelijk idee voor dit onderzoek. Digitalisatie in synthesizer systemen. Helaas kwam kort daarna, in december 2017, VCVRack uit. VCVRack is een modulair sythesizer programma dat nu ook een module op de markt heeft gebracht om hybride systemen te maken. Daarmee kan een fysieke modulaire synthesizer interageren met een virtuele \textit{patch} die in het programma gemaakt is.

Door me te focussen op synthesizers, merkte ik ook dat de doelgroep van het onderzoek voor dit onderzoek te klein zou zijn. Daarom besloot ik om het doelpubliek uit te breiden naar de muzieksector.\newline Toen VCVRack uit de spotlight stapte, merkte ik op dat het maar weinig toegepast werd. Het is een krachtig proof-of-concept dat zeer goedkoop is in vergelijking met de analoge varianten. Waarom zulke ideeën geen doorbraak lijken te hebben in de sector kon het best achterhaald worden door middel van een marktonderzoek.

Vandaar dit marktonderzoek naar de mogelijkheid tot digitale transitie voor de muzieksector.

Ik wil graag Stefaan Samyn, mijn promotor, bedanken om dit onderzoek mogelijk te maken. Ik bedank Thomas Desmedt, mijn co-promotor, voor zijn tijd, raad en toewijding aan het project.\newline 
Ik wil Jo Van Weyenberg, Debora Litzroth en Luna De Leenheir bedanken voor alle steun. Zij daagden mij uit om me uiterst te verdiepen in dit onderzoek en hebben uren besteed aan het nalezen en bekritiseren van deze paper.\newline
Ik wil alle interviewees bedanken voor hun tijd en de informatie die ik van hen kon verkrijgen. Bij deze met name dank aan: Bart Vincent, Thomas Houthave, Adam (Woodie Bundo) Vandenhaute, Saulo Soneghet, Luna Boone, Adam Wilson en Peter Boone. Ik wil de potentiële interviewees bedanken voor hun tijd ookal konden ze helaas niet geïnterviewd worden voor dit onderzoek. Hiervoor dank aan Peter van Praag en Babek Joshghani, Ben Van Camp en Frank Duchêne.\newline Uiteindelijk wil ik ook Debora Litzroth en Mathias Sercu bedanken als tussencontacten voor het verkrijgen van de interviews.

Om te eindigen met een quote van synthesizer designer Don Buchla: \textit{``I've been interested in avant-garde and experimental music far more than I've been interested in traditional form and structure. My instruments have reflected that need.''} Vergelijkbaar hiermee werpt dit onderzoek licht op de mogelijkheden die ontwikkelaars hebben in het betreden van de markt van de muzieksector. Dit in de hoop dat ze correct kunnen inspelen op de moderne noden van artiesten en producers.

% En natuurlijk mijn vriendin Luna die me hier zeer hard in gesteund heeft en heel veel voor me deed toen ik dit schreef. love her sooooo much. 


