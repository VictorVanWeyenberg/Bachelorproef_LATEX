%%=============================================================================
%% Methodologie
%%=============================================================================

\chapter{\IfLanguageName{dutch}{Methodologie}{Methodology}}
\label{ch:methodologie}

%% TODO: Hoe ben je te werk gegaan? Verdeel je onderzoek in grote fasen, en
%% licht in elke fase toe welke stappen je gevolgd hebt. Verantwoord waarom je
%% op deze manier te werk gegaan bent. Je moet kunnen aantonen dat je de best
%% mogelijke manier toegepast hebt om een antwoord te vinden op de
%% onderzoeksvraag.

\section{Inhoud}

Allereerst worden de \textbf{empirische tests} uitgevoerd. Deze tests gaan na of de digitalisatie mogelijk is op technisch vlak. Het doel van dit onderzoek is analoge aparatuur om te vormen in pure software zodat het toegankelijker is voor consumenten met een lager budget. Daarom werd voor de technische tests gebruik gemaakt van een standaard particulieren laptop. De code geschreven voor de uitvoering van de tests wordt samen met de specificaties van het testsysteem respec besproken in sectie \ref{sec:methodologie:empirischetests}.

Zodat de testcases van de empirische tests representatief zouden zijn, werd er gezocht naar real-life usecases uit de muzieksector. Hiervoor zijn interviews afgenomen. De interviews zelf zijn besproken in sectie \ref{sec:interviews}. Hoe de empirische tests opgesteld zijn

\section{Empirische tests}
\label{sec:methodologie:empirischetests}

\subsection{De Testmachine}

\subsection{De Code}

\section{Emotionele tests}

\iffalse \lipsum[21-25] \fi

