
\newcount\foo

\chapter{Testresultaten}
\label{ch:testresultaten}

\section{Testresultaten van Beads}

\foo=1
\loop
  
     \begin{longtable}[c]{|l|l|l|}
        \caption{Beads testresultaten van case \the\foo.}\\
        \hline
        \textbf{Index} & \textbf{\%CPU} & \textbf{\%Mem}\\
        \hline
        \csvreader[
            column count = 50,
            late after line = \\]
        {csvs/beads\the\foo.csv}
        {2=\index,7=\CPUPerc,8=\MemPerc}
        {\index & \CPUPerc & \MemPerc}
    \end{longtable}
  
  \advance \foo +1
\ifnum \foo<7
\repeat

\section{Testresultaten van JASS}

\foo=1
\loop
  
      \begin{longtable}[c]{|l|l|l|}
        \caption{JASS testresultaten van case \the\foo.}\\
        \hline
        \textbf{Index} & \textbf{\%CPU} & \textbf{\%Mem}\\
        \hline
        \csvreader[
            column count = 50,
            late after line = \\]
        {csvs/jass\the\foo.csv}
        {2=\index,7=\CPUPerc,8=\MemPerc}
        {\index & \CPUPerc & \MemPerc}
    \end{longtable}
  
  \advance \foo +1
\ifnum \foo<7
\repeat

\section{Testresultaten van JSyn}

\foo=1
\loop
  
      \begin{longtable}[c]{|l|l|l|}
        \caption{JSyn testresultaten van case \the\foo.}\\
        \hline
        \textbf{Index} & \textbf{\%CPU} & \textbf{\%Mem}\\
        \hline
        \csvreader[
            column count = 50,
            late after line = \\]
        {csvs/jsyn\the\foo.csv}
        {2=\index,7=\CPUPerc,8=\MemPerc}
        {\index & \CPUPerc & \MemPerc}
    \end{longtable}
  
  \advance \foo +1
\ifnum \foo<7
\repeat

