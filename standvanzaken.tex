\chapter{\IfLanguageName{dutch}{Stand van zaken}{State of the art}}
\label{ch:stand-van-zaken}

% Tip: Begin elk hoofdstuk met een paragraaf inleiding die beschrijft hoe
% dit hoofdstuk past binnen het geheel van de bachelorproef. Geef in het
% bijzonder aan wat de link is met het vorige en volgende hoofdstuk.

% Pas na deze inleidende paragraaf komt de eerste sectiehoofding.

\section{Inhoud}

In het hoofdstuk \ref{ch:inleiding}: \nameref{ch:inleiding} werd al gesproken over interviews, methodes voor geluidsgeneratie en libraries voor geluidsverwerking. In dit hoofdstuk zullen alledrie die onderwerpen verder uiteengezet worden. Deze onderwerpen vormen de primaire en een deel van de secundaire bronnen van dit onderzoek.

\section{Interviews}

TO DO:
\begin{itemize}
    \item Push interviews naar git.
    \item Transcribe.
    \item hoofdpunten noteren per interview.
\end{itemize}{}

\subsection{Correspondentie}

\subsection{Bart Vincent}

\subsection{Peter Boone}

\subsection{Thomas Houthave}

\subsection{Vagabundos}

\section{Methodes voor Geluidsgeneratie en -verwerking}

%\textcite{methodes} beschrijft in zijn boek vijf methodes digitale geluidsgeneratie en toepassingen ervan.

\begin{itemize}
    \item Subtractive Synthesis
    \item Additive Synthesis
    \item Granular Synthesis
    \item Wavelet en Corpus-based Synthesis
\end{itemize}{}

De werking van deze methodes wordt hieronder verder toegelicht.

\subsection{Subtractive Synthesis}

Bij subtractive synthesis start de gebruiker met een harmonisch rijke geluidsgolf. Een golf is harmonisch rijk wanneer het veel subfrequenties bevat. Voorbeelden hiervan zijn zaagtand-, blok- en driehoeksgolven Wanneer de golf puur gehoord wordt zal de fundamentele frequentie het dominantst klinken \autocite{harmonics}. Het is de taak van de gebruiker om door middel van frequentiefilters de ongewenste delen van het frequentiespectrum weg te halen en zo het geluid te modelleren \autocite{subtractive}

\textcite{subtractive} bespreken in hun artikel methodes voor digitale geluidsgeneratie en -verwerking. Deze methodes zijn slechts een benadering van wat analoge synthesizers doen. Maar in tegenstelling tot andere pogingen om analoge geluidsgeneratie te digitaliseren zijn de resultaten van deze methodes niet te onderscheiden van het analoge geluid \autocite{subtractive}.

Subtractive synthesis is tot vandaag nog steeds de standaard voor geluidsdesign sinds Robert A. Moog de eerste subtractive synthesizer modules uitbracht in de jaren '60 \autocite{subtractive}. \textcite{guitarpedals} beschrijft in zijn artikel hoe sommige effecten, die vandaag terug te vinden zijn in de effectpedalen voor gitaristen, digitaal verwerkt kunnen worden; allemaal door middel van subtractive synthesis.

\subsection{Additive Synthesis}

Additive synthesis is de tegenhanger van subtractive synthesis. Sinusgolven staan er gekend om maar één frequentie te hebben; hun fundamentele frequentie. Bij additive synthesis gaat de gebruiker tal van sinusgolven bij elkaar optellen om complexere geluiden te bekomen \autocite{additive}.

Voor kleine toepassingen is deze berekening nog haalbaar voor een digitale workstation. Maar van het moment dat de geluiden complexer en harmonisch rijker moeten zijn, loopt de doorlooptijd van de generatie linear op met het aantal sommen van de additieve golf \autocite{additive}.

Wanneer men harmonisch rijke golven wilt genereren zoals de zaagtandgolf, is dit onmogelijk te volbrengen op additieve wijze door het oneindig aantal sommaties dat berekend moet worden \autocite{harmonics}.

\subsection{Granular Synthesis}

Bij granular synthesis start de gebruiker met een of meerdere zeer korte geluidsfragmenten - grains - die aanzien worden als een periode van een geluidsgolf. De gebruiker zal die periode verschillende keren kopiëren en verschalen. De periode krijgt vervolgens ook een aanvangsttijd toegewezen. Door verschillende grains op die manier met elkaar te vermengen, kan de gebruiker het gewenste geluid modelleren \autocite{granular}.

Granular synthesis is van eigen al een techniek die enkel digitaal toegepast kan worden. Met de IT-sector als doelpubliek is het een zeer aantrekkelijke methode voor sound synthesis. De techniek is zeer bruikbaar voor geluidssimulaties en -imitaties, vooral door middel van een AI \autocite{granular}.

\textcite{granular} legt in zijn paper zelfs een manier uit om in real-time (live) aan granulair geluid te genereren. Zelf geeft hij aan het einde van zijn artikel wel toe dat hij geen toekomst ziet voor granular synthesis in de muziekproductie. Dit omdat er teveel parameters zijn die de gebruiker moet beïnvloeden.

\subsection{Wavelet en Corpus-based Synthesis}

Deze twee generatie methodes zijn soorten van granular synthesis. Door hun verschil in implementatie zijn ze elk toepasbaar op andere vlakken.

\subsubsection{Wavelet Synthesis}

Wavelet synthesis biedt de gebruiker niet meerdere maar slechts één periode om het modelleerproces mee te starten. De periode, hier wavelet genoemd, is op voorhand specifiek gekozen door de gebruiker om het gewenste geluid te kunnen bekomen. Verder verloopt het modelleerproces volledig zoals granular synthesis \autocite{wavelet}.

Deze methode wordt meer gebruikt wanneer men zeer gedetailleerd geluid wilt nabootsen. Het modeleerproces vereist dat de gebruiker een goed inzicht heeft in muziektheorie en vergt ook dat het geluid iteratief gemodeleerd wordt 
\autocite{wavelet}. Real-time is het dus slecht toepasbaar.

\subsubsection{Corpus-based Granular Synthesis}

Deze granulaire generatiemethode haalt de grains op uit een databank - de corpus - van vooraf opgenomen geluiden. Deze methode wordt vooral gebruikt wanneer men een specifiek geluid wilt imiteren. Een zoekalgoritme of AI weet dan exact waar hij gelijkaardige grains terug kan vinden en probeert dan het originele geluid na te bootsen aan de hand van die grains 
\autocite{methodes}.

Corpus-based is niet toepasbaar wanneer de gebruiker nieuwe geluiden wilt maken. Niet alleen heeft het dezelfde tekortkomingen als gewone granulaire geluidsgeneratie, het heeft ook nog extra vrijheidsgraden van welke grains te kiezen uit de hele corpus.

\section{Libraries}

\subsection{JMusic}

\subsection{JASS}

\textcite{jass} ontwikkelden JASS (Java Audio Synthesis System) voor hun onderzoek naar efficiente geluidsgeneratie. Deze library is de kers op de taart van zeven voorafgaande onderzoeken naar modellen voor geluidsgeneratie die bedoeld zijn voor geluidseffecten in video games en simulaties. 

\textcite{jass} leggen uit hoe verschillende libraries verschillende doeleinden, implementaties en kosten hebben. Hun doel met JASS is om een library te schrijven die, naast alle requirements die opgelijst staan in hun paper, ook een breed scala van mogelijke toepassingen heeft. JASS is dus bedoeld als all-round library voor om het even welke sector \autocite{jass}.

\subsubsection*{De Code}
\label{sec:jass_code}

JASS kan teruggevonden worden op de website van van den Doel \autocite{jasscode}.

In hun paper leggen \textcite{jass} de innerlijke werking van JASS uit. Hier valt op dat het een modulaire\footnote{Iedere module heeft zijn eigen atomaire rol (oscillator, filter, compressor, versterker etc.). Door ze met elkaar te verbinden - of patchen - krijgt de gebruiker een zelf ontworpen stramien van parameters waarmee hij zijn geluid kan boetseren.} library is. De ontwikkelaar kan zijn eigen modules ontwikkelen door over te erven van gegeven abstracte klassen en interfaces  \autocite{jass}. 

Dit is ook hoe programma's voor geluidsbewerking en (modulaire) synthesizers opgebouwd zijn. Modulaire programma's zijn intuïtief voor gebruikers uit de muzieksector \autocite{bartvincent}.

Een ontwikkelaar instantieert een geluidsgenererende subklasse. Deze klasse erft dus van de \verb+engine.Out+ klasse. Bijgevolg moet deze klasse de \verb+computeBuffer+ methode implementeren. Deze methode kan één van volgende twee dingen doen. Ofwel genereert het een buffer. Ofwel verwerkt het een buffer die de klasse verkrijgt via een \verb+engine.Source+ attribuut. Alle klassen die erven van \verb+engine.In+ en \verb+engine.InOut+ hebben een \verb+addSource+ methode waarmee een \verb+Source+ toegevoegd kan worden \autocite{jass}.

Wanneer een stramien van verwerkte buffers uiteindelijk via een \verb+SourcePlayer+ aangesloten wordt op de audio output van de machine, kan de buffer gehoord worden. Evenals subklassen van \verb+engine.In+ en \verb+engine.InOut+ kan een \verb+SourcePlayer+ meerdere \verb+Source+'s hebben \autocite{jass}.

\subsection{Beads}

Beads is een open-source Java library gemaakt door Oliver Bown. Het project is in 2008 tot stand gekomen met behulp van Monash University in Melbourne en heeft in 2014 zijn laatste update gekregen \autocite{beads}.

De library is bedoeld voor het schrijven van real-time\footnote{Wanneer een library of programma multi-threaded werkt zodat de gebruiker met het geluid kan interageren terwijl het gegenereerd of bewerkt wordt.} programma's voor geluidsverwerking. Java was hier onmiddellijk de taal van voorkeur omdat het open-source is. De ontwikkelaars melden op hun site dat ze een zeer flexibele IO-laag geïmplementeerd hebben zodat de library zich in verschillende contexten kan benutten \autocite{beads}. 

Het doel van Beads is het vergemakkelijken audio-implementatie. Bown en zijn team hopen om de ontwikkeling van audio applicaties toegankelijker te maken voor de standaard programmeur \autocite{beads2}.

\subsubsection*{De Code}

Programmeren in Beads start met het instantiëren van een \verb+AudioContext+. Deze klasse vraagt in zijn constructor naar een audio output device. Het selecteren van een audio output device is mogelijk in native Java \autocite{beadsdocs}.

Wat oscillatoren betreft biedt Beads er reeds een paar voorgeprogrammeerde aan. Deze zijn ondergebracht in de \verb+Buffer+ klasse. De golven kunnen afgespeeld worden aan hand van een \verb+WavePlayer+ klasse \autocite{beadsdocs}.

Beads werkt, net zoals JASS, ook modulair. Klassen die erven van \verb+UGen+ implementeren de \verb+addInput+ methode. De methode vraagt een andere \verb+UGen+ als parameter wiens output de input wordt van de betreffende \verb+UGen+ \autocite{beadsdocs}.

Om \verb+UGen+'s te horen, moeten ze aangesloten worden op de \verb+AudioContext+. De \verb+AudioContext+ kan meerdere \verb+UGen+'s als input aanvaarden. Vervolgens moet de ontwikkelaar de \verb+start+ methode oproepen in \verb+AudioContext+ om het geluid af te laten spelen \autocite{beadsdocs}.

\subsection{JSyn}

JSyn is - de naam verraadt het - een Java library die traditionele modellen van modulaire synthesizers imiteert \autocite{jsyn}. Net zoals Beads is het real-time en open-source verkrijgbaar op GitHub \autocite{jsyngit}.

Mobileer Inc. hebben naast JSyn ook twee andere software audio projecten gereleased: JMSL (Java Music Specification Language) en PortAudio. JMSL is een Java-based specificatie taal waarin gebruikers instrumenten en composities kunnen definiëren. PortAudio is een cross-platform audio IO-library voor C \autocite{jsyn}.

\subsubsection*{De Code}

Beschouw de documentatie van JSyn \autocite{jsyndocs}. Daar vinden we een aantal herkenbare klassen terug zoals basis oscillatoren (sinus-, zaagtand-, driehoeks- en blokgolf), verschillende soorten filters (low-pass, high-pass, band-pass, multi-pole etc.) en zelfs effecten (delay, envelope etc.).

JSyn baseert zich op één moederklasse: \verb+Synthesizer+. Wanneer een ontwikkelaar modules instantieert, moeten die via de \verb+add+ methode toegevoegd worden aan de \verb+Synthesizer+ vooraleer ze gehoord kunnen worden. De \verb+Synthesizer+ klasse staat in voor het starten van alle toegevoegde modules en beslist ook de geluidskwaliteit van de output \autocite{jsyndocs}.

De genererende en verwerkende klassen hebben - net zoals in JASS - een methode die een buffer genereert of verwerkt. Kijk hiervoor terug naar \ref{sec:jass_code}. De methode heet \verb+pullData+ en verkrijgt de te verwerken buffers via de \verb+UnitInputPort+-attributen van de klassen \autocite{jsyndocs}.

JSyn is, evenals Beads en JASS ook modulair. Het verbinden van klassen is mogelijk door de \verb+UnitInputPort+- en \verb+UnitOutputPort+-attributen van genererende en verwerkende klassen. Via de \verb+connect+ methode kan een ontwikkelaar een \verb+UnitOutputPort+ connecteren aan een \verb+UnitInputPort+ \autocite{jsyndocs}.

Een stramien van verwerkte buffers, wordt via de \verb+LineOut+ klasse op de audio output van de machine aangesloten. Wanneer de ontwikkelaar de \verb+start+ methode oproept op zowel de \verb+LineOut+ als de \verb+Synthesizer+, zal het geluid gehoord worden. Het \verb+UnitInputPort+-attribuut van \verb+LineOut+ kan meerdere inputs ontvangen \autocite{jsyndocs}.

\subsection*{In Vergelijking}

\begin{longtable}[c]{l|lll}
         & \textbf{Beads} & \textbf{JASS} & \textbf{JSyn} \\ \hline
        \textbf{Real-time} & Ja & Ja & Ja \\
        \textbf{Modulair} & Ja & Ja & Ja \\
        \textbf{Buffer methode} & calculateBuffer & computeBuffer & pullData \\
        \textbf{Synthesis methode} & Subtractive & Subtractive & Subtractive \\
    \caption{Vergelijking van de drie testlibraries.}
    \label{tab:vergelijking}
\end{longtable}

Wanneer we de documentatie van de drie libraries beter bekijken, valt op dat hun code in essentie hetzelfde doet. Niet alleen hebben ze alle gelijkenissen uit tabel \ref{tab:vergelijking}, de architectuur van het verbinden van genererende en verwerkende buffers is duidelijk ook typerend aan sound libraries.

Uitzonderlijk bij JSyn is wel dat alle parameters een \verb+UnitInputPort+ zijn. Neem een low-pass filter (LPF) als voorbeeld. Gegeven een harmonisch rijke geluidsgolf gaat een LPF alle aanwezige frequenties hoger dan een gegeven cut-off frequentie weg filteren. Het te filteren geluid is hier onze input - zoals bij zowel JSyn als andere libraries het geval is. De cut-off frequentie is hier een parameter voor de module.\newline
Bij de meeste libraries zijn zulke parameters statische getallen. JSyn behandelt zulke parameters als \verb+UnitInputPort+'s \autocite{jsyndocs}.

De filosofie van object-oriented programming is het modelleren van de echte wereld. In dit voorval is JSyn de beste representatie van echte modulaire synthesizers. Parameters van modules zijn maar zelden statische waarden. Artiesten willen een interactieve band hebben met hun geluid. Daarvoor moeten zulke parameters dynamisch aangepast kunnen worden \autocite{vagabundos}. In het geval van modulaire synthesizers \textit{"moduleert"} men parameters door middel van \textit{"control voltage"} \autocite{modular}.

\iffalse

Dit hoofdstuk bevat je literatuurstudie. De inhoud gaat verder op de inleiding, maar zal het onderwerp van de bachelorproef *diepgaand* uitspitten. De bedoeling is dat de lezer na lezing van dit hoofdstuk helemaal op de hoogte is van de huidige stand van zaken (state-of-the-art) in het onderzoeksdomein. Iemand die niet vertrouwd is met het onderwerp, weet nu voldoende om de rest van het verhaal te kunnen volgen, zonder dat die er nog andere informatie moet over opzoeken \autocite{Pollefliet2011}.

Je verwijst bij elke bewering die je doet, vakterm die je introduceert, enz. naar je bronnen. In \LaTeX{} kan dat met het commando \texttt{$\backslash${textcite\{\}}} of \texttt{$\backslash${autocite\{\}}}. Als argument van het commando geef je de ``sleutel'' van een ``record'' in een bibliografische databank in het Bib\LaTeX{}-formaat (een tekstbestand). Als je expliciet naar de auteur verwijst in de zin, gebruik je \texttt{$\backslash${}textcite\{\}}.
Soms wil je de auteur niet expliciet vernoemen, dan gebruik je \texttt{$\backslash${}autocite\{\}}. In de volgende paragraaf een voorbeeld van elk.

\textcite{Knuth1998} schreef een van de standaardwerken over sorteer- en zoekalgoritmen. Experten zijn het erover eens dat cloud computing een interessante opportuniteit vormen, zowel voor gebruikers als voor dienstverleners op vlak van informatietechnologie~\autocite{Creeger2009}.

\lipsum[7-20]

\fi
